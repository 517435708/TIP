\justy{Aktualnie do identyfikacji sesji bądź użytkownika program używa losowego UUID czyli 128 bitowego klucza. Aby zapewnić bezpieczeństwo tego typu rozwiązaniu należy po stronie serwera skonfigurować protokół HTTPS w taki sposób, aby nagłówki wraz z ,,request body'' były szyfrowane (po ewentualnym deployu na cluster kubernetesowy taką czynność może wykonać administrator lub devops). Ponieważ obaj użytkownicy wymieniają się zaszyfrowanymi algorytmem RSA kluczami możliwe, że nie istnieje sposób na odczytanie zapisu rozmowy. Jedyny możliwy atak to pozyskanie pełnego dostępu do komputera jednego z użytkowników (np. za pomocą RATa). Można również próbować odgadnąć klucz składający się z 256 bitów. Istnieje również możliwość wykorzystania ataku Man in the Middle: podpiąć się pod węzeł komunikacyjny pomiędzy użytkownikami, następnie przechwycić moment negocjacji klucza (uprzednio przechwytując publiczne klucze obu użytkowników), wygenerować własne dwa klucze, zaszyfrować je kluczami publicznymi obu użytkowników, a następnie odesłać w ten sposób podrobione wiadomości. Dzięki temu atakujący miałby dostęp do rozmowy.}