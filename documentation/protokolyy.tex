\justy{W tej sekcji opiszemy najważniejsze protokoły zawarte w naszym projekcie.}

\subsection{Dołączanie użytkownika do poczekalni}

\justy{Po uruchomieniu aplikacji użytkownik ma za zadanie wpisać swój pseudonim. Następnie wysyłane jest żądanie do serwera o dołączenie do poczekalni, które składa się z wylosowanego przez klienta klucza publicznego oraz jego pseudonimu. Serwer następnie dodaje użytkownika do poczekalni i nadaje mu unikalny identyfikator, na którym opiera się komunikacja z pozostałymi użytkownikami.}

\subsection{Łączenie użytkowników oraz negocjacje klucza}
\justy{W momencie gdy użytkownik pragnie się połączyć z inną osobą w poczekalni wysyła on żądanie do serwera. Ten tworzy sesję i dołącza do niej użytkownika próbującego się połączyć (inicjującego połączenie). Jeśli osoba odbierająca przyjmie połącznie to serwer dołącza ją do sesji i wysyła odpowiednie komunikaty zawierające klucze publiczne potrzebne do wylosowania 256 bitowego klucza AES po 128 bitów każdy. Obie strony rozmowy wymieniają się zaszyfrowanymi częściami klucza. Po potwierdzeniu otrzymania części klucza przez drugą stronę rozpoczyna się szyfrowana komunikacja między dwoma użytkownikami.}