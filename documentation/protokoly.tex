\subsection{Protokół łączenia użytkowników}
\justy{
W tymże protokole udział bierze dwóch użytkowników zalogowanych oraz serwer, który pośredniczy w przesyłaniu pakietów. Użytkownika żadającego połączenia nazwiemy użytkownikiem A, a użytkownika akceptującego połączenie nazwiemy użytkownikiem B. Na początku użytkownik A i B tworzą parę kluczy RSA, publiczny i prywatny. Klucze publiczne są wysyłane za pomocą serwera między użytkownikami. Następnie użytkownik A tworzy losowy łańcuch składający się z znaków heksadecymalnych o rozmiarze 128 bitów. Następnie szyfruje on ten łańcuch za pomocą klucza publicznego użytkownika B i przesyła go za pomocą serwera. Użytkownik B robi analogiczną operację. Następnie użytkownicy łączą odszyfrowany kluczem prywatnym łańcuch z swoją, wcześniej wygenerowaną cześcią. W taki sposób powstaje 256 bitowy klucz AES, który już jest wykorzystywany do końca rozmowy. Użytkownik A posiada pierwsze 128 bitów klucza, dlatego dołącza otrzymany łańcuch na końcu. Z kolei użytkownik B posiada drugą cześć i dołącza otrzymany łańcuch na początek.} 