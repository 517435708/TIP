Projekt został wykonany z często wspólnymi obowiązkami, podział prac prezentuje się następująco:

\noindent Adrian Golczak: 
\begin{itemize}
	\item nadzorowanie prac projektu,
	\item pisanie dokumentacji projektu,
	\item stworzenie szkieletu serwera oraz klienta,
	\item napisanie metod umożliwiających łączenie się użytkowników z serwerem, poczekalnią oraz samym sobą,
	\item stworzenie ekranów poczekalni, rozmowy.
\end{itemize}
Marcin Kubiak:
\begin{itemize}
	\item pisanie dokumentacji projektu,
	\item generowanie klucza publicznego, prywatnego,
	\item przetwarzanie wysyłanych i odbieranych bajtów.	
\end{itemize}

\subsection{Cele zrealizowane}
\justy{Z pewnością udała nam się negocjacja klucza AES 256 między użytkownikami rozpoczynającymi rozmowę, dodatkowo szyfrowanie i deszyfrowanie bajtów również działa prawidłowo. W trakcie prac udało się uzyskać połączenie między użytkownikami znajdującymi się w tej samej sieci, a sama rozmowa przebiega prawidłowo.}
\subsection{Cele niezrealizowane}
\justy{Nie udało nam się stworzyć aplikacji mobilnej tak jak wcześniej zakładaliśmy, przez co zabrakło czasu na dopracowanie pewnych elementów działającej aplikacji klienckiej opartej na JavaFX.}
\subsection{Napotkane problemy}
\justy{Pisząc aplikację mobilną w pewnym momencie wszystkie funkcjonalności systemu zaczęły generować mnóstwo błędów, żeby to naprawić musielibyśmy od zera cały szkielet kodu aplikacji klienckej zmienić, lepszym sposobem było przejście na aplikację desktopową, gdyż Android Studio sprawiał mnóstwo problemów, a sam emulator zajmował dużo pamięci RAM przez co testowanie aplikacji było bardzo problematyczne i lepszą opcją było przejście na JavaFX.}
\subsection{Perspektywa rozwoju}
\justy{Myślę, że w niedalekiej przyszłości warto zaimplementować algorytm logowania oraz odzyskiwania hasła. Dodatkowo można rozszerzyć funkcjonalność aplikacji klienckiej o czat tekstowy w celu np. wysyłania linków.}