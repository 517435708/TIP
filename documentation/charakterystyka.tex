\justy{Przedmiotem projektu pt. 'Opracowanie bezpiecznego systemu komunikacji głosowej w sieci IP (VoIP) wraz z jego implementacją' jest opracowanie aplikacji mobilnej na urządzenia z systemem Android wyposażonej w algorytmy RSA oraz AES-256 umożliwiające bezpieczną rozmowę pomiędzy dwoma użytkownikami aplikacji. Aplikacja będzie zaprojektowana tak, że osoba trzecia będzie w stanie podsłuchać tylko niewrażliwe dane, dzięki zastosowaniu RSA oraz AES-256. Główną koncepcją projektu jest stworzenie tzw. poczekalni, w której zalogowani użytkownicy bedą mogli się łączyć z kim tylko chcą i odbywać z nim rozmowę. W celu bezpieczeństwa, użytkownicy będą generować klucze publiczne oraz prywatne, które następnie będą używane do szyfrowania, deszyfrowania oraz przesyłaniu klucza szyfru blokowego AES, służącego do szyfrowania rozmowy. Aplikacja ma być łatwa w obsłudze oraz przejrzysta, dlatego będzie ograniczać się tylko do przyjmowania, odrzucania połączenia oraz rozmowy między dwoma użytkownikami. Dodatkowo w jednym pokoju jednocześnie będzie mogło przebywać dwóch klientów, a logowanie do poczekalni nie będzie wymagało podania hasła. Użytkownik będzie mógł kontrolować głośność za pomocą funkcji wbudowanych w telefonie.}