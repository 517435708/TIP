\documentclass{article}
\usepackage{lastpage}
\usepackage[polish]{babel}
\usepackage[T1]{fontenc}
\usepackage[utf8]{inputenc}
\usepackage{graphicx}
\usepackage{fancyhdr}




\newcommand{\version}{v1.0.0}
\newcommand{\lastPage}{5}
\newcommand{\tab}{\hspace{1cm}}

\pagestyle{fancy}
\setlength
\headheight{40pt}
\renewcommand{\footrulewidth}{0.4pt}
\fancyhf{}

\rhead{Dokumentacja projektu}
\lhead{\includegraphics[width=5cm]{images/pp.jpg}\newline Wydział informatyki i telekomunikacji}

\lfoot{Politechnika Poznańska}
\cfoot{Page \thepage\space of\space \lastPage}
\rfoot{19.03.2020r.}

\begin{document}
	\hyphenpenalty=100000
	\exhyphenpenalty=100000
\begin{titlepage}
		\begin{center}
			
						\LARGE
			Politechnika Poznańska
			
			\vspace{0.3cm}
			
			\large
			Wydział informatyki i telekomunikacji
			
			\vspace{3.0cm}
			\huge
			\textbf{Dokumentacja projektu}
			
			\vspace{0.5cm}
			
			\large
			Dokumentacja projektu z zajęć Telefonia IP
			
			\vspace{2.4cm}
			
			\LARGE
			\textbf{Autorzy:}
			
			\vspace{0.3cm}
			
			Adrian Golczak 136239
			adrian.e.golczak@student.put.poznan.pl
			
			\vspace{0.3cm}
			
			Marcin Kubiak 136267
			marcin.w.kubiak@student.put.poznan.pl
			
			\vfill
			
			\normalsize
			Wersja: \version
			
			\vspace{2cm}
			

			
			19.03.2020 r.
			
		\end{center}
\end{titlepage}
\tableofcontents
\newpage
\section{Charakterystyka ogólna projektu}
\tab Przedmiotem projektu pn. 'Opracowanie bezpiecznego systemu komunikacji głosowej w sieci IP (VoIP) wraz z jego implementacją' jest opracowanie aplikacji mobilnej wyposażonej w odpowiednie algorytmy umożliwiające bezpieczną rozmowę pomiędzy dwoma użytkownikami aplikacji. Główną koncepcją projektu jest stworzenie tzw. poczekalni, w której zalogowani użytkownicy bedą mogli się łączyć z kim tylko chcą i odbywać z nim rozmowę. Aplikacja będzie spełniała zasady integralności i poufności, aby uniemożliwić podsłuchanie rozmowy przez osobę trzecią.

\newpage
\section{Architektura systemu}
\subsection{Uproszczona architektura technologii}
\tab Funkcjonalnie, możliwe będzie połączenie dwóch clientów aplikacji mobilnej poprzez server napisany w spring boot framework.\\
\includegraphics[width=12cm]{images/simple architecture.png}
\begin{center}
	\footnotesize
	Rysunek 1. Uproszczona architektura systemu
\end{center}
\subsection{Architektura przepływu}
\includegraphics[width=12cm]{images/flow.png}
\begin{center}
	\footnotesize
	Rysunek 2. Architektura przepływu komunikacji pomiedzy użytkownikiem, klientem, a serwerem.
\end{center}
\section{Wymagania}
W tymże rozdziale skupimy sie na wymaganiach aplikacji z podziałem na funkcjonalne oraz niefunkcjonalne. Opiszemy konkretne wymagania dotyczące aktorów, użytkownika niezalogowanego oraz użytkownika zalogowanego.
\subsection{Funkcjonalne}
Użytkownik niezalogowany: 
\begin{itemize}
\item Podanie pseudonimu
\item Logowanie się do poczekalni
\end{itemize}
Użytkownik zalogowany:
\begin{itemize}
	\item Wysyłanie próśb o połączenie
	\item Akceptacja prośby od drugiego użytkownika
	\item Generowanie klucza publicznego i prywatnego wykorzystując algorytm RSA
	\item Odrzucenie prośby o połączenie
	\item Opuszczenie poczekalni oraz trwającej rozmowy	
\end{itemize}

\subsection{Niefunkcjonalne}
\begin{itemize}
\item łączenie dwóch użytkowników,
\item generowanie ID sesji
\item negocjacje klucza o rozmiarze 256 bitów na potrzeby AES-256,
\item szyfrowanie rozmowy wykorzystując AES-256,
\item minimalna wersja systemu Android: 10.0.0,
\color{red}
\item brak wymogu podania hasła przez użytkownika przy logowaniu,
\item limit użytkowników w poczekalni podyktowany mocą obliczeniową serwera,
\item jedno urządzenie mobilne to jedna instancja klienta,	\item w ramach jednego pokoju może przebywać jedynie dwóch użytkowników,
\item zmiana głośności rozmowy za pomocą wbudowanych funckji sprzętowych urządzenia.
\end{itemize}

\section{Technologie, narzędzia, środowisko, biblioteki, kodeki}

\begin{itemize}
	\item \textit{TeXstudio}
	\item \textit{InteliJ}
	\item \textit{Java11}
	\item \textit{SpringBoot}
	\item \textit{AndroidStudio}
	\item \textit{javax.sound}
	\item \textit{jcodec)}
\end{itemize}

\end{document}
